%Typografia a publikovanie
%Filip Gulán (xgulan00)
%2014

\documentclass[11pt,a4paper,titlepage]{article}
\usepackage[left=2cm,text={17cm,24cm},top=3cm]{geometry}
\usepackage[T1]{fontenc}
\usepackage[czech]{babel}
\usepackage[utf8]{inputenc}
\bibliographystyle{czplain}
\usepackage{times}

\begin{document}

\begin{titlepage}
\begin{center}
    {\textsc{\Huge Vysoké učenie technické v~Brne}}\\
    \smallskip
    {\huge\textsc{Fakulta informačných technológií}}\\
    \bigskip
    \vspace{\stretch{0.382}}    
    \LARGE{Typografia a~publikovanie\,--\,4.\,projekt}\\
    \smallskip
    \Huge{Bibliografické citácie}\\
    \vspace{\stretch{0.618}}
\end{center}
    {\Large 2014 \hfill Filip Gulán }
\end{titlepage}

\section{Úvod}
Typografia svoju kapitolu začala písať v~roku 1444, keď \textit{Johannes Gutenberg} vynašiel kníhtlač. Od tejto doby, aj keď si to neuvedomujeme, je typografia bežnou súčasťou tvorby a~spracovania textu. Významný posun vpred v~oblasti typografie nastal rozmachom osobných počítačov a~príslušného softwarového vybavenia.\cite{Rybicka}


\section{Systém \LaTeX}
\LaTeX \ je typografický sádzací systém, určený na tvorbu nielen odborných a~matematických dokumentov, ale aj jednoduchých dokumentov, alebo rozsiahlych kníh. Systém \LaTeX \ je postavený na formátovacom programe \TeX. \cite{Helmut} 

Už označenie \LaTeX \ namiesto obyčajného latex je reklama sama na seba, ktorá prezentuje svoje schopnosti. Označenie vytvoríme makrom \verb|\LaTeX|. \cite{Spec} 

\section{Komunita \LaTeX}
\LaTeX ová komunita je bezpochyby obrovská ako vo svete, tak aj v~Česku a~na Slovensku. Dôkazom toho sú aj početné stránky, na ktorých sa združujú nadšenci tohoto skvelého systému viz \cite{Tug}.
Komunita sa pravidelne stretáva na rôznych konferencií a~poriadajú sa náučné prezentácie a~semináre z~rôznych oblastí typografie. Jednou z~týchto akcií je aj konferencie s~názvom \textit{TypoBerlín}. \cite{Typob}

Nikoho už asi neprekvapí, že existuje množstvo časopisov, ktoré sa venujú hlavne typografií a~pár ich je dokonca na domácej pôde. Za zmienku stojí napríklad časopis \cite{Typo}.

\section{Typografia na Vysokej škole}
Pri písaní rôznych dokumentov do školy, či už technických správ, alebo dokumentácií k~projektom vzniká potreba napísať prácu čo najrýchlejšie a~najkvalitnejšie. Práve preto veľa študentov siaha práve po \LaTeX e. Vznikli tiež rôzne špecializované školské stránky, určené na to, aby uľahčili prácu so systémom \LaTeX \ viz \cite{Fekt}.

Téma typografie a~jej blízke odvetia nadchla aj početné množstvo študentov, ktorý na túto tému vypracovali diplomovú alebo bakalársku prácu. Za zmienku stojí napríklad bakalárska práca \cite{2d/3d} alebo diplomová práca \cite{Zivyfont}.

\section{Záver}
Typografia a~\LaTeX \ sú bezpochyby zaujímavé odvetvia a~možno práve z~vás sa stane mladý nádejný typograf. Toto odvetvie sa stále vyvíja a~prispôsobuje okolitému svetu. Dôkazom toho môže byť pani \textit{Radana Lencová}, ktorá je známa predovšetkým vývojom nového školského písma \textit{Comenia Script}, ktoré sa práve testuje. \cite{Typor}

\newpage
\renewcommand{\refname}{Referencie}
\bibliography{texty}

\end{document}
